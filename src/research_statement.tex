% !TEX root = ./navarro-cv-en.tex

\section{Research Statement}

 \cvitem{}{ My research has been focused mainly on the estimation of
   uncertainties in the nucleon-nucleon (NN) interaction along with
   the subsequent error propagation into nuclear structure and nuclear
   astrophysics calculations. Even though the study of the NN
   interaction started more than six decades ago, the estimation of
   the co\-rres\-pon\-ding uncertainties has often been overlooked
   until recent years. One of the main reasons behind the lack of
   estimation of errors arising from experimental uncertainties in
   proton-proton (pp) and neutron-proton (np) scattering is probably
   the high level of complexity required to accurately describe the NN
   interaction, specially when all the details of the short-range or
   high momentum effects are taken into account directly. However,
   recent developments in the implementation of microscopically
   derived interactions from quantum chromodynamics (QCD) and a
   constant increase in computing power have renewed interest on the
   quantification of theoretical nuclear uncertainties.}

 \cvitem{}{\quad \quad The research conducted as a part of my thesis
   dissertation was centered on the complete analysis of over eight
   thousand NN scattering experimental data. The main tool that
   allowed such a challenging study was the Delta-Shell (DS)
   potential. This type of interaction represents an implementation of
   a local potential that samples the NN interaction at certain
   concentration radii. The main advantage of a DS potential is that
   it allows probing the relevant contributions of the NN interaction
   depending on the energy range that is being considered. This
   significantly reduces the computational cost while maintaining the
   desired accuracy to describe experimental data. Another important
   aspect in this analysis is the statistical self-consistency in the
   description of the data. Any traditional fit based on a least
   squares procedure makes the fundamental assumption that the
   discrepancies between experimental data and theoretical model,
   collected by the fit's residuals, follow a standard normal
   distribution. Of course this is an assumption that can only be
   confirmed \emph{a posteriori} once the model parameters have been
   determined. However the verification of this assumption is an
   essential step in the propagation of statistical uncertainties into
   any calculation that uses the NN interaction as input. Our analysis
   of NN scattering data paid special attention to the normality of
   the residuals. Several normality tests were used to search for any
   signal of clear deviations from the standard normal distribution
   and none was found. This made our interactions the first family of
   potentials appropriate for nuclear structure calculations with a
   reliable estimate on the statistical uncertainty. With this
   approach it was possible to obtain a new determination of the low
   energy coupling constants that determine the QCD-derived two pion
   exchange potential with credible uncertainties and correlations
   among them. In a similar fashion we have been exploiting our
   validated uncertainty quantification methods to identify signatures
   of charge symmetry breaking in the pion-nucleon coupling
   constant. Even though this constant is present in most realistic NN
   interactions, its value and charge dependence have been a source of
   strong debate throughout the years. Our initial exploration found
   evidence for charge symmetry breaking in the pion-nucleon constant
   at the $1\sigma$ level. Of course, additional experimental data
   could shed better light into this effect.}

 \cvitem{}{\quad \quad As a postdoc at Lawrence Livermore National
   Laboratory (LLNL) my research shifted to nuclear structure with
   density functional theory (DFT) methods. There, I led the
   development of nuclear energy density functionals derived from
   realistic QCD potentials using the density matrix expansion (DME)
   in collaboration with groups at Ohio State University and Michigan
   State University. Although DFT techniques can be applied across the
   full nuclear chart rather successfully, a set of phenomenological
   parameters needs to be adjusted to describe nuclear structure
   properties. On the other hand the direct implementation of
   microscopically derived forces in the form of a nuclear energy
   density functional introduces a great amount of complexities that
   manifest themselves as non-localities. The DME avoids these
   non-localities and allows describing macroscopic properties of
   nuclei. For the actual realization of the DME I implemented new
   capabilities into LLNL's DFT numerical solver HFBTHO, in particular
   finite range interactions that were tested with Gogny
   functionals. The inclusion of new finite range interactions
   introduced the challenge of appropriately}

 \cvitem{}{ managing computational resources. Solving
   this challenge required an efficient use of high performance
   computing, in particular I parallelized the most time consuming
   parts within the complete DFT calculation.}

 \cvitem{}{\quad \quad Another aspect of my research at LLNL was the
   propagation of statistical uncertainties from energy density
   functionals to simulations of astrophysical processes. The main
   input for these simulations are complete mass tables of nuclear
   properties which explore the frontiers of the nuclear chart from
   dripline to dripline. These mass tables are used to calculate
   capture and decay rates and then abundance patterns for heavy
   elements in the solar system can be generated. In order to
   distinguish from different possible scenarios for the origin of
   heavy nuclei and extract meaningful conclusions from these
   abundance patterns, a proper quantification of the underlying
   uncertainties is also needed. To quantify this uncertainties I
   employed a previously generated probability distribution of the
   parameters of a particular functional and generated several mass
   tables to propagate the statistical uncertainties of the functional
   all the way to the abundance patterns. The construction of these
   mass tables required to perform tens of thousands of completely
   independent DFT calculations. In order to perform all of these
   calculations in a reasonable time I implemented a massive MPI
   parallelization of HFBTHO that allowed concurrently calculating the
   properties of hundreds of nuclei. }

 \cvitem{}{\quad \quad My plans for future research here at Ohio
   University and beyond include connecting microscopically-derived
   interactions with large scale astrophysical phenomena and the
   corresponding quantification and propagation of
   uncertainties. QCD-derived interactions will soon be implemented
   for the calculation of heavy neutron-rich nuclei properties. The
   connection between microscopic interactions and astrophysical
   phenomena requires several intermediate steps in which
   uncertainties are present and need to be propagated. While the DME
   has the capacity to connect realistic finite range interactions
   with nuclear properties and several coupling constants within the
   QCD-derived interaction have already been fitted to NN scattering
   data, additional contact interaction terms need to be fitted in
   order to correctly describe nuclear properties. This introduces
   statistical uncertainties that can be quantified and propagated in
   a rather direct fashion by using frequentist techniques. The
   quantification of systematic uncertainties is more delicate, since
   the sources for this type of uncertainty are much more diverse; the
   QCD-derived interactions are subject to specific short-range
   regulators for which a unique form does not exist; several flavors
   of DME are available in the literature and the effect of the order
   in each expansion remains to be explored. Although the
   quantification of systematic uncertainties is not as
   straightforward as the statistical ones, Bayesian techniques can
   provide relevant insight by simultaneously exploring the space of
   available interactions and DME implementations. Once all of these
   uncertainties have been quantified with Bayesian methods, a
   propagation into astrophysical simulations becomes possible. A
   direct Monte-Carlo sampling of the probability distribution of the
   resulting nuclear properties could be used in astrophysical
   calculations to generate error bars in abundance patterns with the
   intent to diagnose the astrophysical site responsible for the
   generation of heavy elements in our solar system.}

 \cvitem{}{\quad \quad My past achievements and research plans align
   well with current priorities in the nuclear physics community.
   Given that several experimental programs, such as the ones at the
   Facility for Rare Isotope Beams (FRIB), will launch operations
   within the next couple of years, it is an ideal time to concentrate
   efforts in uncertainty quantification of theoretical
   predictions. Indeed, the results from these studies will provide
   crucial insight into determining which observables should be
   measured by our experimental colleagues. Finally, a correct
   assessment of uncertainties in nuclear applications is
   indispensable to make a meaningful comparison between different
   models and to subsequently identify the intrinsic nature of
   different nuclear phenomena. }


