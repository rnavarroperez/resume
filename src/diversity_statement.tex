% !TEX root = ./navarro-cv-en.tex

\section{Diversity Statement}

 \cvitem{}{ I have experience working with diverse student populations both as an instructor and as a mentor. As a professor at a Hispanic Serving institution like San Diego State University (SDSU) I have become more familiar with the broad range of unique challenges that students from different backgrounds can experience in higher education. Over 30\% of the physics majors at SDSU are transfer students coming from local community colleges. Transfer students at SDSU are more likely to be students of color, first generation students, and financially disadvantaged students than students who enroll as freshmen. Being a member of an underrepresented group in STEM myself (Hispanic, Latino), one of the only two persons of color, the only Spanish speaking person and the only member of the LGBTQIA+ community among the faculty in the physics department at SDSU, I have noticed that students who also identify as members of an underrepresented group tend to approach me for mentoring, advice and research opportunities more often than other students.}

 \cvitem{}{\quad \quad In addition to providing mentoring through the Cal-Bridge program I have also worked on creating more opportunities for students from underrepresented groups in physics. I currently serve on the Partnership committee for Cal-Bridge, establishing guidelines for graduate programs across the country that are interested in recruiting current and future Cal-Bridge scholars for their programs. More recently, in collaboration with two other faculty members from SDSU, I was awarded a grant from the Department of Energy to develop a traineeship program at SDSU to broaden the participation of students from underrepresented backgrounds in nuclear physics. For the development of both of these efforts I have strongly relied on documents that provide clear and specific guidelines on improving diversity and inclusion in physics. These documents include the TEAM-UP report from the American Institute of Physics, which focuses on increasing the participation of African-American students, the final report of the 2018 task force on diversity and inclusion from the American Astrophysical Society, and the LGBT climate in physics report from the American Physical Society.}

 \cvitem{}{\quad \quad During my time at SDSU I have participated in multiple trainings that have allowed me to learn how to better serve students from different backgrounds. In addition to taking implicit bias training, I have attended training sessions like the Safe Zones@SDSU program, which provides insight into the experiences of students from the LGBTQIA+ community, the Undocually training, which focuses on serving undocumented and mixed status students, the Military Ally Awareness Program, which provides insight into the unique background of the military community, and the Economic Crisis Response Team advocate program, which provides a working understanding of the basic needs of some of our students facing housing or food insecurity.}

 \cvitem{}{\quad \quad All of these experiences and training programs have allowed me to gain greater sensitivity to and a deeper understanding of the diverse challenges that students from different backgrounds can face while attending higher education. As a physics instructor at Mount San Jacinto College I would draw on these and other experiences when working and interacting with students from diverse backgrounds and members of underrepresented groups to provide an equitable environment where all students will have the opportunity to achieve their academic and professional goals.}

 

