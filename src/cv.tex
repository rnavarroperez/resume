% !TEX root = ./navarro-cv-en.tex

\makecvtitle

\section{<Research expertise|Intereses de investigación>}

\cvitem{}{< Nuclear density functional theory, density matrix expansion, heavy nuclei. Effective interactions, Gogny force. Nucleon-Nucleon interaction, chiral potentials. Nuclear structure calculations, few body problems. Uncertainty quantification, error propagation. High performance computing. | Teoría del funcional de densidad nuclear, nucleos pesados. Interacciones effectivas, fuerza de Gogny. Interacción Nucleón-Nucleón, potenciales quirales. Cálculos de estructura nuclear, problemas de pocos cuerpos. Cuantificación de incertidumbres, propagación de errores. Computación de alto rendimineto>}

\section{<Professional experience|Experiencia profesional>}

\entry{
 date = {<since|desde> Aug~2018},
 title = {<Assistant Professor|Profesor Asistente>},
 institution = {San Diego State University},
 address = {<USA|EEUU>},
 %% description = {\ifoption{english}
 %%   {Member of ``Programming Principles, Logic and Verification Group''}
 %%   {Miembro del ``Grupo de Principios de Programación, Lógica y Verificación''}
 %% }
}

%\entry{
%  date = {<Jan|Ene>--Jul 2012},
%  title = {<Lecturer|Catedrático>},
%  institution = {Queen Mary, University of London},
%  address = {UK},
%}

%% \section{<Postdoctoral research|Investigación postdoctoral>}

\entry{
 date = {2017-2018},
 title = {Postdoc},
 institution = {Ohio University},
 address = {<USA|EEUU>},
 description = {\f{Supervisors} Charlotte Elster <and|y> Daniel Phillips},
}

\entry{
 date = {2015--2017},
 title = {Postdoc},
 institution = {Lawrence Livermore National Laboratory},
 address = {<USA|EEUU>},
 description = {\f{Supervisor} Nicolas Schunck},
}

\section{<Teaching experience|Experiencia docente>}

\entry{
 date = {<since|desde> 2019},
 title = {<Computational Physics|Física Computacional>},
 institution = {San Diego State University},
 address = {<USA|EEUU>},
 description = {\f{PHYS 580} <Numerical methods for physics problems with modern Fortran|Métodos numéricos para problemas de física con Fortran moderno>},
}

\entry{
 date = {<since|desde> 2019},
 title = {<Principles of Physics|Principios de Física>},
 institution = {San Diego State University},
 address = {<USA|EEUU>},
 description = {\f{PHYS 195} <Fundamental physics in mechanics|Fundamentos de física en mecánica>},
}

\entry{
 date = {<since|desde> 2018},
 title = {<Classical Mechanics|Mecánica Clásica>},
 institution = {San Diego State University},
 address = {<USA|EEUU>},
 description = {\f{PHYS 350} <Newtonian mechanics with calculus and vectors|Mecánica de Newton con cálculo y vectores>},
}

\entry{
 date = {2018},
 title = {<General Physics|Física General>},
 institution = {Ohio University},
 address = {<USA|EEUU>},
 description = {\f{PHYS 2052} <Classical physics with calculus and vectors|Física clásica con cálculo y vectores>},
}

%\entry{
%  date = {2008--2009},
%  title = {Postdoc},
%  institution = {Max Planck Institute for Software Systems (MPI-SWS)},
%  address = {<Germany|Alemania>},
%  description = {\f{<Research group leader|Lider del grupo>} Andrey Rybalchenko},
%}

\section{<Education|Educación>}

\entry{
  date = {2011--2015},
  title = {<PhD in Physics and Mathematics|Doctorado en Física y Matemáticas>},
  institution = {Universidad de Granada},
  address = {<Spain|España>},
  description = {
    \f{<Thesis|Tesis>} ``Statistical error analysis of the Nuclear Force'' <|\newline(Análisis de los errores estadísticos de la Fuerza nuclear)> \par
    \f{<Supervisors|Supervisores>} Enrique Ruiz Arriola <and|y> José Enrique Amario Soriano
  },
}

\entry{
  date = {2009--2010},
  title = {<Master in Physics and Mathematics|Máster en Física y Matemáticas>},
  institution = {Universidad de Granada},
  address = {<Spain|España>},
  description = {
    \f{<Thesis|Tesis>} ``Solución al Problema de Campos Hipercríticos mediante Condiciones de Frontera'' <\newline(Boundary conditions to solve the hypercritical field problem)|> \par
    \f{Supervisor} Enrique Ruiz Arriola
  },
}

\entry{
  date = {2004--2009},
  title = {<BSc in Physics|Licenciatura en Física>},
  institution = {Universidad de las Américas Puebla},
  address = {<Mexico|México>},
  description = {
    \f{<Thesis|Tesis>} ``Obtención del Estado base para la Cuerda de un Gravitón'' <\newline(Ground state of a graviton string)|> \par
    \f{Supervisor} Roberto Cartas Fuentevilla
  },
}

%\entry{
%  date = {2006--2007},
%  title = {<Abroad studies program|Programa de estudios en el extranjero>},
%  institution = {University of Notre Dame},
%  address = {<USA|EUA>},
%%  description = {
%%    \f{<Thesis|Tesis>} ``Obtención del Estado base para la Cuerda de un Gravitón'' <\newline(Ground state of a graviton string)|> \par
%%    \f{Supervisor} Roberto Cartas Fuentevilla
%%  },
%}

%\ifoption{best papers}{
%  \section{<Selected publications|Publicaciones destacadas>}
%  \biblist{aplas13,pldi11,cav09}
%}{}

\section{<Externally funded projects|Proyectos con financiaión externa>}

\entry{
  date = {2021--2023},
  title = {HADRONS: Helping Amplify Diversity and Research Opportunities in Nuclear Sciences},
  institution = {<Department of Energy| Departamento de Energía>},
  address = {<USA|EEUU>},
  description = {
    \f{<Amount|Presupuesto>} \$200,000<|USD>/2 <years|años>
  },
}

\section{<Prizes and awards|Premios y reconocimientos>}

\award{
  date = {2021},
  title = {Outstanding Faculty in Physics},
  institution = {San Diego State University, College of Sciences},
  address = {<USA|EEUU>},
}


\award{
  date = {2020},
  title = {Faculty Forward Award},
  institution = {San Diego State University},
  address = {<USA|EEUU>},
}

\award{
  date = {2016},
  title = {<Finalist|Finalista>: Research Slam},
  institution = {Lawrence Livermore National Laboratory},
  address = {<USA|EEUU>},
}

\award{
  date = {2015},
  title = {Cum Laude (<PhD|Doctorado>)},
  institution = {Universidad de Granada},
  address = {<Spain|España>},
}

\award{
      date = {Sep -- Nov 2014},
      title = {<PhD. abroad program scholarship|Beca de movilidad para doctorado>},
      institution = {Universidad de Granada},
      address = {<Spain|España>}
}


\award{
  date = {2010--2014},
  title = {<PhD Scholarship|Beca para doctorado>},
  institution = {<National Council on Science and Technology|Consejo Nacional de Ciencia y Tecnología> (CONACyT)},
  address = {<Mexico|México>},
}

\award{
  date = {2009--2010},
  title = {<MSc Scholarship|Beca para máster>},
  institution = {<Carolina Foundation|Fundación Carolina>},
  address = {<Spain|España>},
}

\award{
  date = {2009},
  title = {Magna Cum Laude (<BSc|Licenciatura>)},
  institution = {Universidad de las Américas, Puebla},
  address = {<Mexico|México>},
}

\award{
  date = {2009},
  title = {<Highest GPA (School of Science)|Promedio más alto (Escuela de Ciencias)>},
  institution = {Universidad de las Américas, Puebla},
  address ={<Mexico|México>},
}


\award{
  date = {2006--2007},
  title = {<Member of the Dean's List (School of Science)|Miembro de la Lista del Decano (Escuela de Ciencias)>},
  institution = {Universidad de las Américas, Puebla},
  address ={<Mexico|México>},
}



%\section{Teaching}

%\module{
%  date = {<fall|otoño> 2012},
%  code = {COMP2008},
%  title = {Theory III --- Logic},
%  level = {Undergraduate},
%  students = {50},
%  institution = {UCL},
%}

%\module{
%  date = {<fall|otoño> 2012},
%  code = {COMP1002},
%  title = {Theory I --- Logic},
%  level = {Undergraduate},
%  students = {100},
%  institution = {UCL},
%}

%\module{
%  date = {<spring|primavera> 2012},
%  code = {ELEM007},
%  title = {Intelligent agents and multi-agent systems},
%  level = {Master},
%  students = {15},
%  institution = {QMUL},
%}

\section{<Academic advising|Asesoría académica>}

\subsection{<Thesis supervising|Supervisión de tesis>}

\thesis{
 date = {2019},
 title = {Improving Nuclear Mass Predictions Using Machine Learning Algorithms},
 author = {Garret Gallear},
 degree = {<M.S. in Physics|Máster en Física>},
 institution = {San Diego State University},
 address = {<USA|EEUU>},
}

\subsection{<Undergraduate research supervision|Supervisión de investigación estudiantil a nivel licenciatura>}

\student{
 date = {<Since|Desde> 2021},
 name = {Marielle Duran},
 project = {Monte Carlo Sampling of phenomenological nuclear interactions},
 host = {San Diego State University}
}

\student{
 date = {2020--2021},
 name = {Juan Ortiz},
 project = {Gaussian Process emulation of experimental Nucleon-Nucleon scattering data},
 host = {San Diego State University}
}

\student{
 date = {2020},
 name = {Raul Bernal-Gonzales},
 project = {Re-implementation of the AV18 potential with updated scattering data and uncertainties},
 host = {San Diego State University}
}

\student{
 date = {Since 2019},
 name = {Matthew Crowley},
 project = {Quantification of systematic uncertainty in Nuclear DFT predictions with Random Forests},
 host = {San Diego State University}
}

\student{
 date = {2019--2020},
 name = {Steven Bradley},
 project = {Model independent Nucleon-Nucleon scattering amplitudes using Random Forests},
 host = {San Diego State University}
}

\student{
 date = {2018--2019},
 name = {Diego Fernández del Val},
 project = {Optical Potential model for analyzing elastic scattering data of tightly bound nuclei},
 host = {San Diego State University}
}


\subsection{<Thesis reviewing|Evaluación de tesis>}

\thesis{
 date = {2021},
 title = {Time-series and phasecurve photometry of episodically-active asteroid (6479) Gault in a quiescent state},
 author = {Josiah Nicholas Purdum},
 degree = {<Master of Science in Astronomy|Máster en Ciencias en Astronomía>},
 institution = {San Diego State University},
 address = {<USA|EEUU>},
}

\subsection{<PhD qualifying exam committee|Miembro del comité de examen de candidatura>}

\student{
 date = {<May|Mayo> 2021},
 name = {Stephanie Lauber},
 project = {Benchmarking projected Hartree-Fock as an approximation},
 host = {San Diego State University}
}

\student{
 date = {<March|Marzo> 2021},
 name = {Oliver Gorton},
 project = {Advanced nuclear shell-model calculations for basic and applied science},
 host = {San Diego State University}
}


\student{
 date = {<May|Mayo> 2020},
 name = {Jordan Fox},
 project = {Uncertainty quantification of an empirical shell-model interaction using principal component analysis},
 host = {San Diego State University}
}


%\thesis{
%  date = {2009},
%  title = {Sobre Algunas Clases Polinomiales de Satisfacibilidad Proposicional},
%  author = {José Inácio de Jesus Rodrigues},
%  degree = {<Doctor in Mathematics|Doctor en Matemáticas>},
%  institution = {Universidad de Sevilla},
%  address = {<Spain|España>},
%}

\section{<Invited talks|Conferencias invitadas>}

\entry{
  date = {<Nov|November> 2019},
  title = {Gogny 2019: Microscopic Approaches to Nuclear Structure and Reactions},
  address = {Livermore, CA <USA|EEUU>},
  description = {
    \f{<Title|Título>} ``Recent developments in Microscopically Constrained EDFs'' }
}

\entry{
  date = {<Jan|Enero> 2019},
  title = {42nd Symposium on Nuclear Physics},
  address = {Cocoyoc, Mexico},
  description = {
    \f{<Title|Título>} ``Microscopically based energy density functionals for nuclei using the density matrix expansion'' }
}

\entry{
  date = {<Oct|Octubre> 2018},
  title = {Joint Meeting of the APS Division of Nuclear Physics and the Physical Society of Japan},
  address = {Waikoloa, HI <USA|EEUU>},
  description = {
    \f{<Title|Título>} ``HPC implementation of microscopically constrained energy density functionals'' }
}

\entry{
  date = {<Mar|Marzo> 2018},
  title = {San Diego State University Colloquium},
  address = {San Diego, CA <USA|EEUU>},
  description = {
    \f{<Title|Título>} ``Precision nuclear physics: Uncertainty Quantification in the supercomputing era'' }
}

\entry{
  date = {<Jan|Enero> 2017},
  title = {Workshop on The tower of effective (field) theories and the emergence of nuclear phenomena},
  address = {Saclay <France|Francia>},
  description = {
    \f{<Title|Título>} ``Error quantification and falsification of chiral-EFT interactions'' }
}

\entry{
  date = {<Jan|Enero> 2017},
  title = {American Physical Society April Meeting 2017},
  address = {Washington, DC <USA|EEUU>},
  description = {
    \f{<Title|Título>} ``(Im)precise nuclear forces: From experiment to model'' }
}

\entry{
  date = {<Feb|Febrero> 2016},
  title = {Fairness 2016: Workshop for young scientists with research interests focused on physics at FAIR},
  address = {Garmisch-Partenkirchen, <Germany|Alemania>},
  description = {
    \f{<Title|Título>} ``Statistical uncertainties on the NN interaction and light nuclei'' }
    }

\section{<Professional training|Capacitación profesional>}
\entry{
  date = {<March|Marzo> 2017},
  title = {INT Program},
  institution = {Institute for Nuclear Theory},
  address = {Seattle, WA <USA|EEUU>} ,
  description = {Toward Predictive Theories of Nuclear Reactions Across the Isotopic Chart}
}

\entry{
  date = {<July|Julio> 2016},
  title = {TALENT: Training in Advanced Low Energy Nuclear Theory},
  institution = {University of York},
  address = {<UK|Reino Unido>},
  description = {Density functional theory and self-consistent methods}
}

\entry{
  date = {<June|Junio> 2016},
  title = {INT Program},
  institution = {Institute for Nuclear Theory},
  address = {Seattle, WA <USA|EEUU>} ,
  description = {Bayesian Methods in Nuclear Physics}
}


\section{<Professional service|Servicio profesional>}

 \subsection{<Journal peer reviewing|Evaluador externo en revistas científicas>}

 \committee{
   date = {<since|desde> 2018},
   venue = {PRC: Physical Review C},
 }
  
 \committee{
   date = {2015--2016},
   venue = {IJMPE: International Journal of Modern Physics E},
 }

 \subsection{<Research grant proposal reviewing|Evaluador externo de projectos de investigación>}
 \committee{
   date = {2020},
   venue = {DOE Office of Science Graduate Student Research Program}
 }
 
\subsection{<Diversity, equity and inclusion efforts|Esfuerzos para la diversidad, equidad e inclusión>}

 \committee{
   date = {since 2020},
   venue = {<SDSU College of Sciences Diversity and Inclusion committee member|Miembro del comité de Diversidad en Inclusión de la escuela de ciencias en SDSU> }
 }

 \committee{
   date = {since 2019},
   venue = {<Academic mentor in the Cal-Bridge program|Mentor académico en el programa Cal-Bridge> }
 }

\subsection{<Member of program committee|Miembro del comité del programa>}

\committee{
  date = {2012},
  venue = {<IV Meeting of young researchers of Atomic and Molecular Physics|IV Jornadas de Jóvenes Investigadores en Física Atómica y Molecular>. Universidad de Granada.},
}

%  \subsection{<Conference peer reviewing|Evaluador externo en conferencias>}

%  \committee{
%    date = {2009, 2013},
%    venue = {VMCAI: Verification, Model Checking, and Abstract Interpretation},
%  }

%  \subsection{<Other reviewing service|Otros servicios de evaluación>}
%  
%  \committee{
%    date = {2011},
%    venue = {<National postdoctoral internships funded by CONACyT in Mexico|Programa nacional de estancias posdoctorales financiadas por CONACyT en México>},
%  }

%  \subsection{<Technical editor|Editor técnico>}
%  
%  \committee{
%    date = {2002},
%    venue = {Proceedings of the 7th International Conference on Parametric Optimization and Related Topics},
%  }

\section{<Volunteer work|Trabajo voluntario>}

  \entry{
    date = {2018},
    title = {<Judge|Juez>},
    organisation = {<Distric Science Day|Día de la Ciencia>},
    address = {Ohio University, <USA|EEUU> }
  } 

  \entry{
    date = {2007},
    title = {<Staff|Colaborador>},
    organisation = {<XII Regional meeting on Physics Reasearch and Teaching|XII Encuentro regional de Investigación y Enseñanza de la Física>},
    address = {UDLAP, <Mexico|México> }
  } 
  
  \entry{
    date = {2006},
    title = {<Organiser|Organizador>},
    organisation = {<X Conference Cycle in Physics and Mathematics|X Ciclo de Conferencias de Física y Matemáticas>},
    address = {UDLAP, <Mexico|México> }
  }
  
  \entry{
    date = {2005--2006},
    title = {<Treasurer|Tesorero>},
    organisation = {<Student Council, Physics and Mathematics board|Consejo Estudiantil, mesa de Física y Matemáticas>}, 
    address = {UDLAP, <Mexico|México> }
  }

\section{<Skills and competences|Habilidades personales>}

\subsection{<Languages|Idiomas>}

\skill{
  subject = {<English|Inglés>},
  details = {<Fluent (Written and spoken)|Fluido (Hablado y escrito)>},
}

\skill{
  subject = {<Spanish|Español>},
  details = {<Native|Nativo>},
}

\skill{
  subject = {<German|Alemán>},
  details = {<Basic (Spoken)|Básico (Hablado)>},
}

\subsection{<Programming languages|Lenguajes de programación>}

\programming{
  subject = {<Imperative|Imperativos>},
  details = {Fortran},
}

\programming{
 subject = {<Object oriented|Orientado a objetos>},
 details = {Python},
}

\programming{
  subject = {<Parallel computing|Computación en paralelo>},
  details = {OpenMP, MPI},
}

%\programming{
%  subject = {<Logic and functional|Lógicos y funcionales>},
%  details = {Prolog, OCaml, Haskell},
%}

\programming{
  subject = {<Symbolic and numerical|Análisis simbólico y numérico>},
  details = {Mathematica, Matlab},
}

\programming{
  subject = {<Document typesetting|Composición de documentos>},
  details = {\LaTeXTeX},
}

\programming{
  subject = {<Web design|Diseño web>},
  details = {HTML, CSS},
}

%\programming{
%  subject = {<Databases|Bases de datos>},
%  details = {SQLite, MySQL},
%}

%\programming{
%  subject = {<Social media|Medios sociales>},
%  details = {<Twitter and Facebook apps|Apps en Twitter y Facebook>},
%}

\subsection{<General interests|Intereses generales>}

\cvitem{}{<Physics, mathematics, science, high performance computing, critical thinking and technology.|Física, matemáticas, ciencia, pensamiento crítico y tecnología.>}


\section{<References|Referencias>}

\reference{
  name = {Charlotte Elster},
  institution = {<Physics and Astronomy Department|
                  Departamento de Física y Astronomía>. Ohio University},
  address = {Clippinger 265, Athens OH-45701, <USA|EEUU>},
  phone = {+1~(740)~593~1697},
  email = {elster@ohio.edu},
}

\reference{
  name = {Nicolas Schunck},
  institution = {<Nuclear Data and Theory Group|
                  Grupo de Datos y Teoría Nuclear>. Lawrence Livermore National Laboratory},
  address = {L-414, P.O. Box 808, Livemore CA-94551, <USA|EEUU>},
  phone = {+1~(925)~422~1621},
  email = {schunck1@llnl.gov},
}


\reference{
  name = {Enrique Ruiz Arriola},
  institution = {<Atomic, Molecular and Nuclear Physics Department. University of Granada|
                  Departamento de Física Atómica Molecular y Nuclear. Universidad de Granada>},
  address = {Av. Fuentenueva S/N. 18001 Granada, <Spain|España>},
  phone = {+34~(958)~246~170},
  email = {earriola@ugr.es},
}

%% \reference{
%%   name = {José Enrique Amaro Soriano},
%%    institution = {<Atomic, Molecular and Nuclear Physics Department. University of Granada|
%%                    Departamento de Física Atómica Molecular y Nuclear. Universidad de Granada>},
%%   address = {Av. Fuentenueva S/N. 18001 Granada, <Spain|España>},
%%   phone = {+34~(958)~240~028},
%%   email = {amaro@ugr.es},
%% }

%% \reference{
%%   name = {Andreas Nogga},
%%   institution = {<Institute for Nuclear Physics|Instituto de Física Nuclear> (IKP). Forschungszentrum J\"ulich},
%%   address = {Wilhelm-Johnen-Strasse. 52425 J\"ulich. <Germany|Alemania>},
%%   phone = {+49~(2461)~61~4725},
%%   email = {a.nogga@fz-juelich.de},
%% }

%% \reference{
%%   name = {Eduardo Garrido Bellido},
%%   institution = {<Institute for the Structure of Matter|Insituto de Estructura de la Materia>. Consejo Superior de Investigaciones Científicas},
%%   address = {C. Serrano, 113  Madrid. <Spain|España>},
%%   phone = {+34~(915)~941~134},
%%   email = {e.garrido@iem.cfmac.csic.es},
%% }

