% !TEX root = ./navarro-cv-en.tex

%-----       letter       ---------------------------------------------------------
%recipient data 
\recipient{Department of Physics}{San Diego State University.\\ San Diego, CA. USA.}
\date{\today}
\opening{Hiring Committee,}
\closing{Sincerely yours,}
\enclosure[Attached]{curriculum vit\ae{}, list of publications, teaching statement and research statement}          % use an optional argument to use a string other than "Enclosure", or redefine \enclname
\makelettertitle

With over six years of experience in theoretical nuclear physics,
strong communication and interpersonal skills as well as in-depth
knowledge of high performance computing, I am a strong candidate for
the Assistant Professor position in Theoretical Nuclear Physics at San
Diego State University. I recently started a postdoctoral position in
the Department of Physics and Astronomy at Ohio University. My current
research focuses on uncertainty quantification in nuclear
interactions and nuclear many body systems.

%\quad

I earned my PhD from the University of Granada, Spain on February of
2015. As a part of my doctoral research I undertook the most ambitious
statistical analysis up to date of over eight thousand nucleon-nucleon
scattering data.  This analysis allowed, for the first time, the
quantification of statistical uncertainties and an objective
assessment of the predictive power of phenomenological and
microscopically derived interactions. I started my first postdoctoral
appointment in May of 2015 at Lawrence Livermore National Laboratory
(LLNL). There, I expanded my research skills to include the structure
of heavy nuclei using density functional theory (DFT) methods. I led
the computational implementation of energy density functionals derived
from realistic interactions and worked on the propagation of
statistical uncertainties from nuclear functionals to simulations of
the astrophysical processes responsible for the creation of heavy
elements. Such work could provide clues to identify the
characteristics of the astrophysical site were the elements in our
solar system were formed. This work required developing new
capabilities into the DFT code used at LLNL. In particular, I
implemented a new hybrid parallel programming module to do large scale
simulations.

%\quad

Although I have spent most of my career doing research in theoretical
nuclear physics, I feel a strong aspiration to transmit my
appreciation for the physical sciences to students. I first became
aware of this aspiration while tutoring high school students as a
college senior in Mexico. That experience gave me direct glimpse into
the learning process that my students were going through, from
encountering a complex subject to fully internalizing and
understanding its more fundamental components, and ultimately applying
this newly acquired knowledge to solve problems. While the original
terms of my position at Ohio University did not include teaching, I
specifically negotiated the opportunity to instruct a physics
course. As a result of this negotiation next spring I will be
instructing the General Physics course.  
%
%\quad
%
%% I am convinced that teaching in higher education gives a unique
%% opportunity to impart scientific knowledge while at the
%% same time engage them with the deep understanding of the world around
%% us that the study of physics allows.
%% I firmly believe that teaching
%% science will allow me to contribute to a society that appreciates
%% rational and critical thinking.
In addition to being able to teach
introductory level classes, my work experience in nuclear theory,
statistical methods and high performance computing allows me teaching
classes in nuclear theory and computational physics.

During the American Physical Society April meeting this year, I
participated on the roundtable \emph{Improving the Climate in Physics
for LGBT+ Physicists}. Through the discussions in this session,
personal experiences as a gay physicist and reading the 2016 APS
report \emph{LGBT Climate in Physics} I have learned about some the
challenges that LGBT physicists, along with people of other
underrepresented groups, experience in higher education. Some of these
challenges include feelings of isolation, expectation of closeted
behavior and difficulties in identifying role models. I intend to
continue engaging in these important issues, and as a faculty member I
would work to mitigate these and other challenges faced by LGBT
individuals in higher education through active participation in
programs like SafeZones@SDSU.

%\quad

I am convinced that my combination of communication skills, research
experience and commitment to diversity and excellence make me an
ideal addition the Department of Physics faculty at San Diego State
University. Please find attached a copy of my CV, a full list of
publications, teaching statement and research statement. Thank you for
your time and consideration. I look forward to hearing from the hiring
committee.

%\quad

\makeletterclosing


