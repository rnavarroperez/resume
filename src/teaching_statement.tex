% !TEX root = ./navarro-cv-en.tex

\section{Teaching Statement}

 \cvitem{}{\quad \quad As an undergraduate student in a university
   with a small physics bachelor's program and no graduate program in
   physics I learned the value of having opportunities to experience
   the scientific research process. Most of my upper-division
   theoretical classes involved reading recent journal articles,
   understanding them with the help of the instructor, and presenting
   them to the rest of the class. Laboratory classes were meant to
   simulate the discovery process that occurs in research laboratories
   and therefore did not require following step by step instructions,
   with expected results, printed on a lab manual. Instead, each team
   was assigned a physical phenomenon and the corresponding equipment
   to study it. Then, each team was responsible for designing an
   experiment, presenting it to the rest of the class so that all the
   teams could perform it, and the results had to be presented in the
   form of an article. Of course, the teacher was present during the
   whole process providing guidance and assistance whenever the
   students needed it. Recently I've come to learn that this type of
   active learning is known as inquiry based learning. This is the
   type of environment that I intend to provide my students with, one
   where scientific inquiry is the main engine that drives the
   learning process. This experience during my undergraduate studies
   has been crucial not only in my career development as a researcher
   but also in allowing me to approach every challenge with a rational
   and critical attitude towards making important decisions.}

 \cvitem{}{\quad}

 \cvitem{}{\quad \quad I am also aware that teaching can go well
   beyond the classroom and research opportunities for students. Now
   that I am settled in my new position as a postdoc at Ohio
   University I recently signed up to volunteer as an OHIO First
   Scholar Advocate. The purpose of this program is to provide
   support, guidance and encouragement to Ohio University's
   first-generation college students as they learn to navigate the
   sometimes confusing and intimidating college environment. As a
   native Spanish speaker who participated in an exchange program with
   the University of Notre Dame as an undergraduate student I am
   familiar with some of the barriers and challenges that Hispanic and
   other ethnic minorities students can face when attending university
   in the United States. As a professor I would draw on these and
   other experiences when working and interacting with students from
   diverse backgrounds and members of underrepresented groups.}

  \cvitem{}{\quad}

 \cvitem{}{\quad \quad As teacher in the physical sciences I want to
   achieve three main goals with my students: That the students
   acquire a scientific mindset, that they develop problem-solving
   skills by applying newly learned concepts, and that they improve
   their communication skills in order to explain complex ideas in
   terms that can be understood by their peers. The latter two are
   real life skills that will later become necessary in their
   professional careers, whether they decide to pursue graduate
   studies or go into industry, government or teaching careers. How to
   achieve these goals will depend on the type of course. Introductory
   level courses benefit from-lecture driven classes with derivations
   on the blackboard and homework assignments in which the concepts
   introduced in class are used to solve different kinds of written
   problems.  In contrast, for upper-division and graduate courses I
   plan to assign projects that allow the students to familiarize
   themselves with current research. In these projects the students
   will apply the same methods of this research to solve a simpler,
   yet interesting, problem and write a report in the style of a
   journal article. For introductory, upper division and graduate
   courses I expect my students to make at least one short oral
   presentation to the rest of the class in which they explain an
   example of a practical application of the class' subjects.}

 \pagebreak

 \cvitem{}{\quad }
 

 \cvitem{}{\quad \quad Just as the pedagogical approach to achieve the
   main goals of applying a scientific mindset, solving problems and
   communicating ideas are different for different courses, the
   methods to assess if such goals have been met also need to be
   tailored to the type of course. I will evaluate introductory level
   courses with a mixture of homework assignments, examinations, and
   oral presentations. Upper and graduate level courses will be
   evaluated based on participation during class discussions, homework
   assignments, oral presentations and a final project that the
   students will work on throughout the whole course.}

\cvitem{}{\quad }
 
 \cvitem{}{\quad \quad While I want research to be an integral part of
   my teaching, I also want my teaching to complement my
   research. This means creating opportunities to involve students in
   my research projects in nuclear and computational physics,
   encouraging them to submit their original work for publication in
   peer reviewed journals. These opportunities provide the students
   with hands-on experiences that enhance the skills they acquire in
   the classroom and also prepare them to solve different kinds of
   problems inside and outside an academic setting.}



