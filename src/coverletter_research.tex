% !TEX root = ./navarro-cv-en.tex

%-----       letter       ---------------------------------------------------------
%recipient data
\recipient{Department of Physics}{University of Somewhere.\\ Somewhere USA.}
\date{\today}
\opening{Hiring Committee,}
\closing{Sincerely yours,}
\enclosure[Attached]{curriculum vit\ae{}, list of publications and research statement}          % use an optional argument to use a string other than "Enclosure", or redefine \enclname
\makelettertitle

With over five years of experience in theoretical nuclear physics
research, in-depth knowledge of high performance computing, and strong
communication and interpersonal skills, I am a strong candidate for
the [position] at [institution]. I have been a post-doc in the Nuclear
Data and Theory group at Lawrence Livermore National Laboratory under
the guidance of Nicolas Schunck since May of 2015. My current research
focuses on uncertainty quantification in nuclear interactions and
nuclear structure using frequentist and Bayesian techniques.

\quad

I earned my PhD from the University of Granada, Spain under the
supervision of Enrique Ruiz Arriola and Jose Enrique Amaro Soriano on
February of 2015. As a part of my doctoral research I undertook the
most ambitious statistical analysis to date of over eight thousand
proton-proton and neutron-proton scattering data. This analysis
allowed, for the first time, the quantification of statistical
uncertainties in nucleon-nucleon potentials. This is a very important
milestone towards objectively evaluating the predictive power of both
phenomenological and microscopically derived interactions.

\quad

As a post-doc at LLNL, I have expanded my research portfolio to
include the structure of heavy nuclei using density functional theory
(DFT). In particular, I have led the computational implementation of
energy density functionals derived from realistic interactions using
the density matrix expansion method (DME). This work will allow
connecting the properties of heavy nuclei with the microscopic
characteristics of an interaction derived from quarks and gluons as
degrees of freedom. I have also applied my knowledge of uncertainty
quantification to the propagation of statistical uncertainties of
nuclear energy density functionals to calculations of nuclear
structure properties used in simulations of the r-process. This
analysis will help identifying the characteristics of the
astrophysical site were the elements in our solar system were
formed. This work required developing, verifying and validating new
capabilities in the LLNL DFT solver, such as the implementation of
finite range potentials and the development of a hybrid parallel
programming module.

\quad

For my future research I plan to apply QCD-derived interactions to the
calculation of astrophysical nuclear phenomena with the corresponding
uncertainty quantification. The DME framework will soon allow
calculating properties of heavy nuclei and such properties can then be
used as input for astrophysical simulations like the
r-process. [Something related to the department's interests].

\quad


I am convinced that my strong research background, commitment to
excellence and interest in new developments on high performance
computing will make me an ideal addition to the [research group's
name] at [institution]. Please find attached a copy of my CV, a full
list of publications and a research statement. Thank you for your time
and consideration. I look forward to hearing from the hiring
committee.

\quad

\makeletterclosing
